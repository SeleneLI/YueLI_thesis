%*******************************************************************************
%****************************** Seventh Chapter *********************************
%*******************************************************************************

\chapter{Conclusion and Future Work}
\label{cha:conclusion}
In this chapter, we summarize our major contributions and discuss future research directions.

\section{Major Contributions}

\subsection{Evaluation of LISP Mapping System}
Describe the main observations of Chapter.~\ref{cha:mds_evaluation}.

\subsection{LISP-Views: LISP Mapping System Monitor}
Describe the main observations of Chapter.~\ref{cha:LISPViews}.

\subsection{Assessing LISP interworking performance through RIPE Atlas}
Describe the main observations of Chapter.~\ref{cha:pxtr}.

\subsection{ns-3 Implementation of LISP and LISP-MN}
Describe the main observations of Chapter.~\ref{cha:ns-3}.



\section{Discussion \& Future work}

\begin{itemize}[noitemsep,topsep=0pt]
    \item Evaluation of LISP Mapping System (Chapter.~\ref{cha:mds_evaluation})
    \begin{itemize}[noitemsep,topsep=0pt]
        \item Evaluate the latest \emph{Stability} and \emph{Consistency} performance of LISP mapping system. 
        \item Compare to the results in Chapter.~\ref{cha:mds_evaluation} and get LISP developing trend.
        \item Synchronized mechanism should be proposed to the MDS.
    \end{itemize}
    
    \item LISP-Views: LISP Mapping System Monitor (Chapter.~\ref{cha:LISPViews}).
    \begin{itemize}[noitemsep,topsep=0pt]
        \item The implementation of a REST API for LISP-Views.
        \item The deployment of LISP-Views on multiple VPs.
        \item Test of IPv6 behavior for LISP-Views.
    \end{itemize}
    
    \item Assessing LISP interworking performance through RIPE Atlas (Chapter.~\ref{cha:pxtr}).
    \begin{itemize}[noitemsep,topsep=0pt]
        \item Evaluate LISP IPv6 performance for LISP-Lab probe.
        \item Compare LISP IPv6 performance between LIP6 probe connecting to PETR with MPLS or not.
        \item In our experiment, as LIP6 probe does not receive any \emph{ping} responses for 42 IPv4 and 10 IPv6 destinations, we plan to cluster these destinations by geography and explore the reasons. Besides, we try to modify the configurations so that PxTR of LISP-Lab allows to receive the IPv6 traffic from LISP-Lab probe. The aim is to evaluate thoroughly IPv6 performance of LISP-Lab platform. Furthermore, as the degradation on performance observed in our paper is mainly due to the PITR selection by the destinations. It will be interesting to explore more the interaction between BGP anycast and LISP interworking. 
    \end{itemize}
    
    \item ns-3 Implementation of LISP and LISP-MN (Chapter.~\ref{cha:ns-3}).
    \begin{itemize}[noitemsep,topsep=0pt]
        \item Comparison the performance between only host supporting LISP, only router supporting LISP and both of them supporting LISP (LISP double encapsulation).
        \item Comparison the performance between SMR and Map-Versioning.
    \end{itemize}
\end{itemize}