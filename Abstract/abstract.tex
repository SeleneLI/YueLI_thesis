% ************************** Thesis Abstract *****************************
% Use `abstract' as an option in the document class to print only the titlepage and the abstract.
\begin{abstract}
The \emph{\acrfull{lisp}} is proposed in 2006 to initially address Internet scalability issues. It is based on a map-encap mechanism to split the \emph{who} and the \emph{where} of the current IP addresses. To retrieve the association between them, a new network entity called the (\emph{\acrfull{mds}}) is introduced. Although \acrshort{lisp} is currently under standardization in IETF and is deployed in the wild by two testbeds at the same time, it is still young. It lacks the thorough measurement works to show its realistic performance in the large-scale networks and to improve itself. 

%In this dissertation, we assess \acrshort{lisp} from the different aspects :
%\begin{inparaenum}[(i)]
%	\item the measurements on the \acrshort{mds}, including: evaluating its performance, and proposing a comprehensive monitor to supervise it;
%	\item the assessment of \acrshort{pxtr}.
%	\item the evaluation of \acrshort{lisp} mobility, comprising: numerically analyzing in different scenarios, and implementing \acrshort{lisp} mobility extensions on an existing simulator.
%\end{inparaenum}
%
%We continuously measured the \acrshort{mds} for seventeen days. The results show that it is stable and consistent over most of time. Nevertheless, instability and inconsistency are observed although they are rare events. We define a new taxonomy to classify them so to facilitate the deeper analysis. As these instability and inconsistency have never been observed, we propose a more comprehensive LISP monitoring architecture to dynamically supervise the whole \acrshort{mds}. After one full month of testing, our proposed monitor provides more mapping information, and can show the \acrshort{mds} performance by various metrics. We also evaluate \acrshort{lisp} interworking performance with legacy Internet by conducting two experiments in the different years. Both results show that \acrshort{lisp} is stable although the stretch of \acrshort{pxtr} is introduced. The \acrshort{pxtr} brings the negative effects for the short trips, but can be ignored for the intercontinental long-distance transmission. Besides, the position of \acrshort{pxtr} either near to the sources or the destinations can decrease the latency a lot. Moreover, we numerically analyze the mobility performance of \acrshort{lisp} in three scenarios. We calculate the low bounds of handover for the different components implementing \acrshort{lisp}. Each scenario has its own advantages and shortcomings. Besides, to facilitate the researchers to verify their proposals or test the new features on \acrshort{lisp} mobility, we implement \acrshort{lisp} mobility extensions under ns-3.27 on an existing simulator.

In this dissertation, we assess \acrshort{lisp} from the different aspects :
\begin{inparaenum}[(i)]
	\item the measurements on the \acrshort{mds}: we continuously measured the \acrshort{mds} for seventeen days. The results show that it is stable and consistent over most of time. Nevertheless, instability and inconsistency are observed although they are rare events. We define a new taxonomy to classify them so to facilitate the deeper analysis.
	\item Proposing a comprehensive monitor to supervise the \acrshort{mds}: as these instabilities and inconsistencies have never been observed, we propose a more comprehensive LISP monitoring architecture to dynamically supervise the whole \acrshort{mds}. After one full month of testing, our proposed monitor provides more mapping information, and can show the \acrshort{mds} performance by various metrics.
	\item The assessment of \acrshort{lisp} interworking performance with legacy Internet: we evaluate it by conducting two experiments in the different years. Both results show that \acrshort{lisp} is stable although a stretch is introduced. The interworking brings the negative effects for the short trips, but can be ignored for the intercontinental long-distance transmissions. % Besides, the position of \acrshort{pxtr} either near to the sources or the destinations can decrease the latency a lot.
	\item The evaluation of \acrshort{lisp} mobility: we analyze the \acrshort{lisp} mobility performance in three scenarios. We estimate the handover delay and overhead for the different components implementing \acrshort{lisp}. Each scenario has its own advantages and shortcomings. Besides, to facilitate the researchers to verify their proposals or test the new features on \acrshort{lisp} mobility, we implement \acrshort{lisp} mobility extensions under ns-3.27 on an existing simulator.
\end{inparaenum}
\end{abstract}